\documentclass{article}
\usepackage{bartabs}
\usepackage{url}
\usepackage{verbatim}
\usepackage{float}
\floatstyle{plaintop}
\restylefloat{table}
\usepackage[justification=justified,singlelinecheck=false,tableposition=top]{caption}
\captionsetup[table]{labelfont=bf,labelsep=newline}
\usepackage{cprotect}
\usepackage{booktabs}

\newcommand{\bt}{{\tt bartabs}}
\renewcommand{\tablename}{Example}

\title{The \bt{} package}
\author{Justin J. Meza \\ \url{justin.meza@gmail.com}}

\begin{document}

\maketitle

\begin{abstract}
This package provides support for generating small, horizontal bar data graphics, in the spirit of those discussed starting on p.~123 of Edward Tufte's {\it The Visual Display of Quantitative Information} (Cheshire, Connecticut, 2007).  For example, here is a bar set at 50\% of the default width with error bars set at 25\%: \mbarerr{50}{0.25}.
\end{abstract}

\section{Getting the package}

The \bt{} package is hosted in a repository on GitHub at \url{github.com/justinmeza/bartabs}.  If you have {\tt git} installed, you can check out your own copy with the command:

\begin{verbatim}
  git clone git@github.com:justinmeza/bartabs.git
\end{verbatim}

\noindent
There, you will find the style file as well as this example document.

\section{Using the package}

Once you obtain the style file, you must declare that you are using it in your document by adding the following line before \verb+\begin{document}+:

\begin{verbatim}
  \usepackage{bartabs}
\end{verbatim}

\noindent
Doing so will let you use several commands to create bar data graphics, described next in examples.

\begin{table}[H]
\begin{tabular}{p{0.5\textwidth}p{0.5\textwidth}}
\verb+Lorem \mbar{50} ipsum.+ & Lorem \mbar{50} ipsum. \\
\verb+Lorem \mbar{3.14159} ipsum.+ & Lorem \mbar{3.14159} ipsum. \\
\end{tabular}
\caption{A solid bar of a percentage of the default width.}
\end{table}

\begin{table}[H]
\begin{tabular}{p{0.5\textwidth}p{0.5\textwidth}}
\verb+Lorem \mbarnorm{50}{100} ipsum.+ & Lorem \mbarnorm{50}{100} ipsum. \\
\verb+Lorem \mbarnorm{50}{75} ipsum.+ & Lorem \mbarnorm{50}{75} ipsum. \\
\end{tabular}
\caption{A solid bar of a normalized fraction of the default width.}
\end{table}

\begin{table}[H]
\begin{tabular}{p{0.5\textwidth}p{0.5\textwidth}}
\verb+Lorem \mbarerr{50}{0.5} ipsum.+ & Lorem \mbarerr{50}{0.5} ipsum. \\
\verb+Lorem \mbarerr{100}{1.0} ipsum.+ & Lorem \mbarerr{100}{1.0} ipsum. \\
\end{tabular}
\caption{A solid bar of a percentage of the default width with left and right error lines of a fraction of that width.}
\end{table}

\begin{table}[H]
\begin{tabular}{p{0.5\textwidth}p{0.5\textwidth}}
\verb+Lorem \mbarnormerr{50}{100}{0.5} ipsum.+ & \\
& Lorem \mbarnormerr{50}{100}{0.5} ipsum. \\
\verb+Lorem \mbarnormerr{50}{75}{0.618} ipsum.+ & \\
& Lorem \mbarnormerr{50}{75}{0.618} ipsum. \\
\end{tabular}
\caption{A solid bar of a normalized fraction of the default width with left and right error lines of a fraction of that width.}
\end{table}

\begin{table}[H]
\setlength{\mbarwidth}{8em}
\begin{tabular}{p{0.5\textwidth}p{0.5\textwidth}}
\verb+\setlength{\mbarwidth}{8em}+ & \\
\verb+Lorem \mbar{50} ipsum+ & Lorem \mbar{50} ipsum. \\
\verb+Lorem \mbar{3.14159} ipsum.+ & Lorem \mbar{3.14159} ipsum. \\
\end{tabular}
\cprotect\caption{The default width of the bars can be controlled with the \verb+\mbarwidth+ length.}
\end{table}

\begin{table}[H]
\setlength{\mbarheight}{3ex}
\begin{tabular}{p{0.5\textwidth}p{0.5\textwidth}}
\verb+\setlength{\mbarheight}{3ex}+ & \\
\verb+Lorem \mbar{50} ipsum.+ & Lorem \mbar{50} ipsum. \\
\verb+Lorem \mbar{3.14159}} ipsum.+ & Lorem \mbar{3.14159} ipsum. \\
\end{tabular}
\cprotect\caption{The default height of the bars can be controlled with the \verb+\mbarheight+ length.}
\end{table}

\begin{table}[H]
\setlength{\mbarerrlinelen}{2pt}
\begin{tabular}{p{0.5\textwidth}p{0.5\textwidth}}
\verb+\setlength{\mbarerrlinelen}{2pt}+ & \\
\verb+Lorem \mbarerr{50}{0.5} ipsum.+ & Lorem \mbarerr{50}{0.5} ipsum. \\
\verb+Lorem \mbarerr{100}{1.0} ipsum.+ & Lorem \mbarerr{100}{1.0} ipsum. \\
\end{tabular}
\cprotect\caption{The default thickness of the error line can be controlled with the \verb+\mbarerrlinelen+ length.}
\end{table}

\begin{table}[H]
\begin{tabular}{p{0.5\textwidth}p{0.5\textwidth}}
\verb+Lorem \mbarerrl{50}{0.5} ipsum.+ & Lorem \mbarerrl{50}{0.5} ipsum. \\
\verb+Lorem \mbarerrl{100}{0.95} ipsum.+ & Lorem \mbarerrl{100}{0.95} ipsum. \\
\end{tabular}
\caption{A solid bar of a normalized fraction of the default width with a left error line a fraction of that width.}
\end{table}

\begin{table}[H]
\begin{tabular}{p{0.5\textwidth}p{0.5\textwidth}}
\verb+Lorem \mbarerrr{50}{0.25} ipsum.+ & Lorem \mbarerrr{50}{0.25} ipsum. \\
\verb+Lorem \mbarerrr{100}{0.05} ipsum.+ & Lorem \mbarerrr{100}{0.05} ipsum. \\
\end{tabular}
\caption{A solid bar of a normalized fraction of the default width with a right error line a fraction of that width.}
\end{table}

\begin{table}[H]
\begin{tabular}{p{0.5\textwidth}p{0.5\textwidth}}
\verb+Lorem \mbarerrlr{50}{0.5}{0.25} ipsum.+ & \\
& Lorem \mbarerrlr{50}{0.5}{0.25} ipsum. \\
\verb+Lorem \mbarerrlr{100}{0.95}{0.05} ipsum.+ & \\
& Lorem \mbarerrlr{100}{0.95}{0.05} ipsum. \\
\end{tabular}
\caption{A solid bar of a percentage of the default width with right and left error lines separate fraction of that width.}
\end{table}

\begin{table}[H]
\begin{tabular}{p{0.5\textwidth}p{0.5\textwidth}}
\verb+Lorem \mbarnormerrl{50}{100}{0.5} ipsum.+ & \\
& Lorem \mbarnormerrl{50}{100}{0.5} ipsum. \\
\verb+Lorem \mbarnormerrl{50}{75}{0.618} ipsum.+ & \\
& Lorem \mbarnormerrl{50}{75}{0.618} ipsum. \\
\end{tabular}
\caption{A solid bar of a normalized fraction of the default width with a left error line of a fraction of that width.}
\end{table}

\begin{table}[H]
\begin{tabular}{p{0.5\textwidth}p{0.5\textwidth}}
\verb+Lorem \mbarnormerrr{50}{100}{0.5} ipsum.+ & \\
& Lorem \mbarnormerrr{50}{100}{0.5} ipsum. \\
\verb+Lorem \mbarnormerrr{50}{75}{0.618} ipsum.+ & \\
& Lorem \mbarnormerrr{50}{75}{0.618} ipsum. \\
\end{tabular}
\caption{A solid bar of a normalized fraction of the default width with a right error line of a fraction of that width.}
\end{table}

\begin{table}[H]
\begin{tabular}{p{0.5\textwidth}p{0.5\textwidth}}
\verb+Lorem \mbarnormerrlr{50}{100}{0.5}{0.25} ipsum.+ & \\
& Lorem \mbarnormerrlr{50}{100}{0.5}{0.25} ipsum. \\
\verb+Lorem \mbarnormerrlr{50}{75}{0.95}{0.05} ipsum.+ & \\
& Lorem \mbarnormerrlr{50}{75}{0.95}{0.05} ipsum. \\
\end{tabular}
\caption{A solid bar of a normalized fraction of the default width with right and left error lines separate fraction of that width.}
\end{table}

\begin{table}[H]
\begin{verbatim}
\begin{center}
  \begin{tabular}{lrl}
    \toprule
      Name & \multicolumn{2}{c}{Datum} \\
    \midrule
      Lorem & $10\pm0.1$ & \mbarerr{10}{0.1} \\
      Ipsum & $20\pm0.2$ & \mbarerr{20}{0.2} \\
      Dolor & $40\pm0.3$ & \mbarerr{40}{0.3} \\
      Sit   & $80\pm0.4$ & \mbarerr{80}{0.4} \\
    \bottomrule
    \end{tabular}
\end{center}
\end{verbatim}
\begin{center}
  \begin{tabular}{lrl}
    \toprule
      Name & \multicolumn{2}{c}{Datum} \\
    \midrule
      Lorem & $10\pm0.1$ & \mbarerr{10}{0.1} \\
      Ipsum & $20\pm0.2$ & \mbarerr{20}{0.2} \\
      Dolor & $40\pm0.3$ & \mbarerr{40}{0.3} \\
      Sit   & $80\pm0.4$ & \mbarerr{80}{0.4} \\
    \bottomrule
    \end{tabular}
\end{center}
\caption{Bars can be used to provide a simple illustration of scientific data in a table.}
\end{table}

\section{To Do}

\begin{itemize}

\item Right now, the source code is incredibly ugly.  It needs to be cleaned up and made more modular.

\item Colored bars and error lines may be interesting to implement.

\end{itemize}

\end{document}
